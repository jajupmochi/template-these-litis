%% Copyleft 2018 Jean-Baptiste Louvet
%% Copyright (C) 2014 Dorian Depriester
%% http://blog.dorian-depriester.fr
%%
%% This file may be distributed and/or modified under the conditions
%% of the LaTeX Project Public License, either version 1.3c of this
%% license or (at your option) any later version. The latest version
%% of this license is in:
%%
%%    http://www.latex-project.org/lppl.txt
%%
%% and version 1.3c or later is part of all distributions of LaTeX
%% version 2006/05/20 or later.
%%
%% This work has the LPPL maintenance status `maintained'.
%%
%% The Current Maintainer of this work is Dorian Depriester
%% <contact [at] dorian [-] depriester [dot] fr>.
%%
%% This is main.tex for French PhD Thesis.
%% See http://blog.dorian-depriester.fr/latex/template-these/template-complet-pour-manuscrit-de-these for help


%%%%%%%%%%%%%%%%%%%%%%%%%%%%%%%%%%%%%%%%%
%           Fichier maitre				%
%%%%%%%%%%%%%%%%%%%%%%%%%%%%%%%%%%%%%%%%%

% \documentclass[a4paper, french, 11pt, final, twoside,openright]{report}
\documentclass[a4paper, french, 11pt, draft, twoside,openright]{report}
\input{Preambule}		% Liste des packages et de leurs options
\input{CommandesPerso}	% Commandes et environnements perso
\input{PageDeGarde}


% Méta-données du PDF
\hypersetup{
    pdfauthor={Auteur},
    pdfsubject={Manuscrit de thèse de doctorat},
    pdftitle={La meilleure manière d'aller d'un point $a$ à un point $b$},
    pdfkeywords={TODO}%TODO
}


%%%%%%%%%%%%%%%%%%%%%%%%%%%%%%%%%%%%%%%%%%%%%%%%%%%%%%%%%%%%%%%%%
%%   			Liste des fichiers à compiler					%
%%%%%%%%%%%%%%%%%%%%%%%%%%%%%%%%%%%%%%%%%%%%%%%%%%%%%%%%%%%%%%%%%
%	\includeonly{Chapitre1,Chapitre2,Annexes}


% Répertoire contenant les images
\graphicspath{{Images/}}
% Infos de la page de garde
\author{Un super \textsc{Doctorant}}
\title{La meilleur manière d'aller d'un point $a$ à un point $b$}
\specialite{Informatique}
\ecole{la meilleure école du monde}
\date{\today}
\directeur{Guru Directeur de thèse}
\encadrant{Encadrant aspirant Guru} %Optionnel
\jurya{Civilité / prénom \textsc{Nom}}{Grade / fonction / statut / lieu d'exercice}{Rapporteur ou examinateur ou directeur de thèse ou codirecteur de thèse}
\juryb{Civilité / prénom \textsc{Nom}}{Grade / fonction / statut / lieu d'exercice}{Rapporteur ou examinateur ou directeur de thèse ou codirecteur de thèse}
\juryc{Civilité / prénom \textsc{Nom}}{Grade / fonction / statut / lieu d'exercice}{Rapporteur ou examinateur ou directeur de thèse ou codirecteur de thèse}
\juryd{Civilité / prénom \textsc{Nom}}{Grade / fonction / statut / lieu d'exercice}{Rapporteur ou examinateur ou directeur de thèse ou codirecteur de thèse}
\jurye{Civilité / prénom \textsc{Nom}}{Grade / fonction / statut / lieu d'exercice}{Rapporteur ou examinateur ou directeur de thèse ou codirecteur de thèse}

\newcommand{\ver}{0.0}% Version du document
\version{\ver}% À commenter si l'on ne veut pas afficher la version et date de compilation sur la page de garde

%%%%%%%%%%%%%%%%%%%%%%%%%%%%%%%%%%%%%%%%%%%%%%%%%
% Ajoute un timestamp de compilation dans le footer du document 
% À commenter dans la version finale
\fancyfoot[L]{Compilation du \today{} à \currenttime{} – version \ver}
\pagestyle{fancy}
\begin{document}
% Préambule
	\pagenumbering{roman}
	\pagedegarde
	\cleardoublepage
		% Table des matières
			\setcounter{tocdepth}{2}	% Pas besoin de trop détailler le sommaire ici (chapitres/sections/sous-sections)
			\dominitoc						% Génération des mini-toc	\pagenumbering{arabic}
			\tableofcontents
		% Liste des figures
			\renewcommand*\listfigurename{Liste des figures}
			\listoffigures
		% Liste des tableaux
		\listoftables

%%%%%%%%%%%%%%%%%%%%%%%%%%%%%%%%%%%%%		
%        Contenu du document        %
%%%%%%%%%%%%%%%%%%%%%%%%%%%%%%%%%%%%%
	\setcounter{mtc}{3}	% "Corrige" les minitocs décallés à cause des chapter* (ex : table des matières)
	\pagenumbering{arabic}
	%%%%%%%%%%%%%%%%%%%%%%%%%%%%%%%%%%%%%%%%%%%%%%%%%%%%%%%%%%%%%%%%%%%%%%%%%%%%%%%%%%%%%%%%%%%%%
%%									Introduction											%
%%%%%%%%%%%%%%%%%%%%%%%%%%%%%%%%%%%%%%%%%%%%%%%%%%%%%%%%%%%%%%%%%%%%%%%%%%%%%%%%%%%%%%%%%%%%%
\chapter*{Introduction}
	\citationChap{
	Au commencement était le verbe,\\
	et le verbe était conjugué,\\
	et le verbe était la conjugaison.
	}{Jean Jean}

%%%%%%%%%%%%%%%%%%%%%%%%%%%%%%%%%%%%%%%%%%%%%%%%%%%%%%%%%%%%%%%%%%%%%%%%%%%%%%%%%%%%%%%%%%%%%


Ce template a pour but de vous aider à mettre en forme votre manuscrit de thèse.
Il est un dérivé d'un template de Dorian \textsc{Depriester} (docteur de l'ENS des Mines de Paris, oklm).
Son blog est disponible à l'adresse suivante: \href{http://blog.dorian-depriester.fr/latex/template-these/template-complet-pour-manuscrit-de-these}{http://blog.dorian-depriester.fr}, je vous prie de lire attentivement l'article y présentant le présent document.
Il comporte plein d'options bien utiles que je vous recommande d'explorer.
De nombreux paquets sont chargés par défaut dans le fichier \verb|Preambule.tex|, je vous conseille d'aller y jeter un œil pour voir tout l'attirail que vous avez à votre disposition.
Deux paquets bien utiles sont présents: \verb|showkeys| et \verb|todonotes|.
Des liens pointant vers leurs manuels sont donnés en commentaires à côté de leurs appels dans \verb|Preambule.tex|, je vous conseille chaudement d'y jeter un œil.
Le premier affiche explicitement dans le document compilé les \verb|label| et leurs utilisations.
Tout cela est visible à la fin de cette phrase et dans la suite de ce document et notamment l'introduction du chapitre~\ref{ch1}.
Le second paquet permet de mettre des notes pour trop tard dans son manuscrit, de manière à y revenir par la suite, à l'aide de la commande \verb|\todo{}|.
Je vous recommande l'utilisation de l'option \verb|inline| qui permet de faire un encadré sur tout une ligne, comme montré ici:

\todo[inline]{Améliorer cette introduction en l'enrichissant avec des examples et une référence biblique supplémentaire.}

Attention de nombreuses commandes ne fonctionnent pas dans l'environnement \verb|todo|.

De manière à accélérer sa compilation, et pour faciliter son développement et debuggage, ce document est par défaut en mode \verb|draft| (attention, les images ne sont pas affichées dans ce mode).
Ce document est donné de base en mode \verb|draft|.
Pour quitter le mode \verb|draft|, il faut soit le supprimer des options de la commande \verb|\documentclass| du fichier \verb|Manuscrit.tex|, soit le remplacer par \verb|final| (c'est la ligne commentée juste au-dessus du \verb|\documentclass|.
Le fait de passer en mode \verb|final| désactive \verb|showkeys| mais pas \verb|todonotes|.

Pour faciliter la compilation de ce document, un \verb|Makefile| est à votre disposition.
La commande \verb|make| dans le répertoire du rapport le compile et donne un pdf contenant la date de compilation.
Par exemple: \verb|Manuscrit_JBL_2018_02_07_15:39:42.pdf|.
Le \verb|Makefile| est à modifier pour mettre votre propre nom dans le nom du fichier compilé (première ligne, variable \verb|NAME|).
Un \verb|make clean| nettoie le répertoire de compilation de tous les résidus de compilation, y compris les manuscrits compilés.

Pour faciliter la gestion des différentes versions de votre manuscrit de thèse, deux bouts de code ont été ajoutés dans \verb|Manuscrit.tex|, juste avant le \verb|\begin{document}|:

\begin{verbatim}
\newcommand{\ver}{0.0}
\version{\ver}
\end{verbatim}

Ce code permet d'indiquer sur la page de garde le numéro de version (ici 0.0, mais peut être n'importe quelle chaîne de caractères, ou formule comme \verb|\newcommand{\ver}{$\alpha$}|), à faire évoluer au fur et à mesure de l'écriture de votre chef d'œuvre. Dans la version finale, commenter ces deux lignes pour enlever la mention d'une version sur la page de garde.

Le second bout de code est le suivant:

\begin{verbatim}
\fancyfoot[L]{Compilation du \today{} à \currenttime{} – version \ver}
\pagestyle{fancy}
\end{verbatim}

Il inclut la version et la date de compilation dans le pied de page, de manière à savoir sur quelle version du document on est en train de travailler sans avoir à imprimer systématiquement la page de garde.
Attention, ce second bout de code ne fonctionne pas si l'on commente le premier, il faut donc commenter les deux à la fois (ou seulement ce second).

	%%%%%%%%%%%%%%%%%%%%%%%%%%%%%%%%%%%%%%%%%%%%%%%%%%%%%%%%%%%%%%%%%%%%%%%%%%%%%%%%%%%%%%%%%%%%%
%%									Chapitre 1											%
%%%%%%%%%%%%%%%%%%%%%%%%%%%%%%%%%%%%%%%%%%%%%%%%%%%%%%%%%%%%%%%%%%%%%%%%%%%%%%%%%%%%%%%%%%%%%
\chapter{Au commencement était le début}\label{ch1}
	\citationChap{
	The thing about quotes on the internet is that you can not confirm their validity
	}{Abraham Lincoln}
	\minitoc

%%%%%%%%%%%%%%%%%%%%%%%%%%%%%%%%%%%%%%%%%%%%%%%%%%%%%%%%%%%%%%%%%%%%%%%%%%%%%%%%%%%%%%%%%%%%%



Tout commença par le début.
Rien d'autre.
Ensuite arriva l'après-début.
Puis l'avant-milieu.
À un moment, le milieu s'est imposé.
La suite, je vous la laisse deviner.

Nous verrons au début, en section~\ref{sect1:premiere} une première section.
Ensuite, le milieu arrive plus vite que prévu en section~\ref{sect1:deuxieme}.
Nous y discuterons du bon usage de \enquote{seconde} et \enquote{deuxieme}.
La fin est très précipitée car la section~\ref{sect1:ccl} amène une première conclusion sur ce travail.
\clearpage

\section{Une première section}\label{sect1:premiere}
    Les jambons sont cuisinés\footnote{Mais j'aime pas ça}.
    \todo[inline]{Cuisiner le reste du repas.}

\section{Une deuxième Section}\label{sect1:deuxieme}
    S'il n'y a que deux sections, il faut dire \enquote{seconde} en non pas \enquote{deuxième}.
    Vive la France.

\section{Conclusion}\label{sect1:ccl}
    À faire avant la fin de ma thèse:
    \begin{itemize}
        \item reprendre une alimentation normale;
        \item voir la lumière du jour;
        \item inviter cette fille à boire un verre;
        \item rédiger mon manuscrit;
        \item dormir plus de 18h par semaine;
        \item libérer l'Ardèche.
    \end{itemize}

	%%%%%%%%%%%%%%%%%%%%%%%%%%%%%%%%%%%%%%%%%%%%%%%%%%%%%%%%%%%%%%%%%%%%%%%%%%%%%%%%%%%%%%%%%%%%%
%%									Chapitre 2												%
%%%%%%%%%%%%%%%%%%%%%%%%%%%%%%%%%%%%%%%%%%%%%%%%%%%%%%%%%%%%%%%%%%%%%%%%%%%%%%%%%%%%%%%%%%%%%

\chapter{On ne lit pas une thèse comme un roman}
	\citationChap{
            En gros, on a trois doctorants donc trois cartouches.
	}{Laurent Vercouter}
	\minitoc

%%%%%%%%%%%%%%%%%%%%%%%%%%%%%%%%%%%%%%%%%%%%%%%%%%%%%%%%%%%%%%%%%%%%%%%%%%%%%%%%%%%%%%%%%%%%%

Le texte introductif d'un chapitre peut-être placé juste en-dessous de la mini table des matières, ou sur la page suivante.
Tout dépends de là où l'on place la commande \verb|\clearpage| dans le code.
J'aime bien les intros qui sont placées sous la mini table des matières, et suivies d'un saut de page.
C'est donc ce que je fais ici.
\clearpage

\section{Comment lire une thèse}
    Il faut procéder de la manière suivante:
    \begin{itemize}
        \item lire l'intro;
        \item lire la conclusion;
        \item comprendre ni l'un ni l'autre;
        \item lire la contribution principale;
        \item s'arrêter au milieu de sa lecture car on s'endort;
        \item reprendre par l'intro;
        \item abandonner la lecture;
        \item de toute façon, cette thèse était écrite sous Word.
    \end{itemize}

    Le protocole pour rédiger une thèse sous Word est simple: apprendre à utiliser \LaTeX.
    Le bénéfice principal de l'usage de \LaTeX est la possibilité de s'adresser aux autres d'une manière particulièrement méprisante et prétentieuse~\cite{Louvet17}.
    Faire du \LaTeX, c'est être \enquote{in}.
    C'est croire que l'on fait partie du gratin, alors que l'on est simplement gratiné.
    Mais on peut se la raconter avec ses polices vectorielles et son approche typographique stricte.

    \subsection{En plus concis}
        \LaTeX: bien. Word: pas bien.
        
\section{Conclusion}
    Le monde est habité par des esprits maléfiques qui tentent sournoisement de détourner toute rigueur typographique.
    Il faut lutter contre ce fléau.
    Dans la composition d'un document comme une thèse de doctorat, la sobriété est de mise.
    On est pas sous paint ou à coder du CSS.
    Pour mettre en valeur une partie de son texte il ne faut pas utiliser de commande exotique ou de mise en forme brute.
    Pas de \verb|\textbf|, de \verb|textit| et encore moins de \verb|\underline|.
    Une commande est à votre disposition pour toute mise en avant de votre prose: \verb|\emph|.
    Cela met délicatement en valeur les parties \emph{qui sont importantes}.

	%%%%%%%%%%%%%%%%%%%%%%%%%%%%%%%%%%%%%%%%%%%%%%%%%%%%%%%%%%%%%%%%%%%%%%%%%%%%%%%%%%%%%%%%%%%%%
%%									Conclusion											%
%%%%%%%%%%%%%%%%%%%%%%%%%%%%%%%%%%%%%%%%%%%%%%%%%%%%%%%%%%%%%%%%%%%%%%%%%%%%%%%%%%%%%%%%%%%%%
\chapter{Tout est fini}
	\citationChap{
	Je sais que cette fois, c'est la fin\\
        Je sais que l'on n'y peut plus rien\\
        Je sais que je devrais t'oublier,\\
        Que je ne devrais pas pleurer,\\
        Que je ne devrais pas crier\\
        Mais je t'aime, je t'aime,\\
        Je t'aime, t'aime,\\
        T'aime, t'aime,\\
        T'aime
	}{Claude François}
	\minitoc

%%%%%%%%%%%%%%%%%%%%%%%%%%%%%%%%%%%%%%%%%%%%%%%%%%%%%%%%%%%%%%%%%%%%%%%%%%%%%%%%%%%%%%%%%%%%%



L'heure est arrivée de conclure.
C'est terminé.
J'ai passé un super moment avec vous.
La vie est belle.
Vous êtes quelqu'un de bien.
Tout est allé si vite.
J'ai eu de la chance de vous rencontrer.
Ces moments passés à vos côtés m'ont laissé un souvenir fort, qui m'accompagnera toute la vie.

La séparation est toujours une expérience difficile.
C'est dur de se dire que tout va changer du jour au lendemain.
Que rien ne sera plus pareil.
Que le monde va basculer autour de moi.
La douleur nous rappelle que nous sommes vivants.
Et vulnérables.

Je sais que je vais être mis en difficulté.
De nombreux défis se profilent devant moi.
Parfois je serai ébranlé.
Mais, si je tombe, toujours je me relèverai.
Car j'ai maintenant en moi la force de soulever des montagnes.
C'est ça que je vous dois.
Je suis invincible.

\hfill Merci.

\vfill Prenez soin de vous, et des autres.

	
	% Annexes
	\begin{appendix}
		\pagenumbering{Roman}
                %%%%%%%%%%%%%%%%%%%%%%%%%%%%%%%%%%%%%%%%%%%%%%%%%%%%%%%%%%%%%%%%%%%%%%%%%%%%%%%%%%%%%%%%%%%%%
%%									ANNEXES 												%
%%%%%%%%%%%%%%%%%%%%%%%%%%%%%%%%%%%%%%%%%%%%%%%%%%%%%%%%%%%%%%%%%%%%%%%%%%%%%%%%%%%%%%%%%%%%%
\chapter{Annexes}

%%%%%%%%%%%%%%%%%%%%%%%%%%%%%%%%%%%%%%%%%%%%%%%%%%%%%%%%%%%%%%%%%%%%%%%%%%%%%%%%%%%%%%%%%%%%%


\section{Figures annexes}
Des figures ici.

\section{Tableaux annexes}
Des tableaux là.

	\end{appendix}

      \bibliographystyle{alpha}
      \bibliography{Manuscrit.bib}
	
\end{document}
