%%%%%%%%%%%%%%%%%%%%%%%%%%%%%%%%%%%%%%%%%%%%%%%%%%%%%%%%%%%%%%%%%%%%%%%%%%%%%%%%%%%%%%%%%%%%%
%%									Chapitre 2												%
%%%%%%%%%%%%%%%%%%%%%%%%%%%%%%%%%%%%%%%%%%%%%%%%%%%%%%%%%%%%%%%%%%%%%%%%%%%%%%%%%%%%%%%%%%%%%

\chapter{On ne lit pas une thèse comme un roman}
	\citationChap{
            En gros, on a trois doctorants donc trois cartouches.
	}{Laurent Vercouter}
	\minitoc

%%%%%%%%%%%%%%%%%%%%%%%%%%%%%%%%%%%%%%%%%%%%%%%%%%%%%%%%%%%%%%%%%%%%%%%%%%%%%%%%%%%%%%%%%%%%%

Le texte introductif d'un chapitre peut-être placé juste en-dessous de la mini table des matières, ou sur la page suivante.
Tout dépends de là où l'on place la commande \verb|\clearpage| dans le code.
J'aime bien les intros qui sont placées sous la mini table des matières, et suivies d'un saut de page.
C'est donc ce que je fais ici.
\clearpage

\section{Comment lire une thèse}
    Il faut procéder de la manière suivante:
    \begin{itemize}
        \item lire l'intro;
        \item lire la conclusion;
        \item comprendre ni l'un ni l'autre;
        \item lire la contribution principale;
        \item s'arrêter au milieu de sa lecture car on s'endort;
        \item reprendre par l'intro;
        \item abandonner la lecture;
        \item de toute façon, cette thèse était écrite sous Word.
    \end{itemize}

    Le protocole pour rédiger une thèse sous Word est simple: apprendre à utiliser \LaTeX.
    Le bénéfice principal de l'usage de \LaTeX est la possibilité de s'adresser aux autres d'une manière particulièrement méprisante et prétentieuse~\cite{Louvet17}.
    Faire du \LaTeX, c'est être \enquote{in}.
    C'est croire que l'on fait partie du gratin, alors que l'on est simplement gratiné.
    Mais on peut se la raconter avec ses polices vectorielles et son approche typographique stricte.

    \subsection{En plus concis}
        \LaTeX: bien. Word: pas bien.
        
\section{Conclusion}
    Le monde est habité par des esprits maléfiques qui tentent sournoisement de détourner toute rigueur typographique.
    Il faut lutter contre ce fléau.
    Dans la composition d'un document comme une thèse de doctorat, la sobriété est de mise.
    On est pas sous paint ou à coder du CSS.
    Pour mettre en valeur une partie de son texte il ne faut pas utiliser de commande exotique ou de mise en forme brute.
    Pas de \verb|\textbf|, de \verb|textit| et encore moins de \verb|\underline|.
    Une commande est à votre disposition pour toute mise en avant de votre prose: \verb|\emph|.
    Cela met délicatement en valeur les parties \emph{qui sont importantes}.
