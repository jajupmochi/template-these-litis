%%%%%%%%%%%%%%%%%%%%%%%%%%%%%%%%%%%%%%%%%%%%%%%%%%%%%%%%%%%%%%%%%%%%%%%%%%%%%%%%%%%%%%%%%%%%%
%%									Chapitre 1											%
%%%%%%%%%%%%%%%%%%%%%%%%%%%%%%%%%%%%%%%%%%%%%%%%%%%%%%%%%%%%%%%%%%%%%%%%%%%%%%%%%%%%%%%%%%%%%
\chapter{Au commencement était le début}\label{ch1}
	\citationChap{
	The thing about quotes on the internet is that you can not confirm their validity
	}{Abraham Lincoln}
	\minitoc

%%%%%%%%%%%%%%%%%%%%%%%%%%%%%%%%%%%%%%%%%%%%%%%%%%%%%%%%%%%%%%%%%%%%%%%%%%%%%%%%%%%%%%%%%%%%%



Tout commença par le début.
Rien d'autre.
Ensuite arriva l'après-début.
Puis l'avant-milieu.
À un moment, le milieu s'est imposé.
La suite, je vous la laisse deviner.

Nous verrons au début, en section~\ref{sect1:premiere} une première section.
Ensuite, le milieu arrive plus vite que prévu en section~\ref{sect1:deuxieme}.
Nous y discuterons du bon usage de \enquote{seconde} et \enquote{deuxieme}.
La fin est très précipitée car la section~\ref{sect1:ccl} amène une première conclusion sur ce travail.
\clearpage

\section{Une première section}\label{sect1:premiere}
    Les jambons sont cuisinés\footnote{Mais j'aime pas ça}.
    \todo[inline]{Cuisiner le reste du repas.}

\section{Une deuxième Section}\label{sect1:deuxieme}
    S'il n'y a que deux sections, il faut dire \enquote{seconde} en non pas \enquote{deuxième}.
    Vive la France.

\section{Conclusion}\label{sect1:ccl}
    À faire avant la fin de ma thèse:
    \begin{itemize}
        \item reprendre une alimentation normale;
        \item voir la lumière du jour;
        \item inviter cette fille à boire un verre;
        \item rédiger mon manuscrit;
        \item dormir plus de 18h par semaine;
        \item libérer l'Ardèche.
    \end{itemize}
